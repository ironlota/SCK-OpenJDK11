\subsubsection{Introduction}

\subsubsubsection{Virtualization}

Virtualization is used throughout the world and one of growing trend for hosting services within data-centers. \cite{rachamalla_mishra_kulkarni_2013}
Virtualization uses dynamic resource allocation and migration techniques. Memory over-commit is one of commonly used technique to facilitate server consolidation where the total memory size for all running VMs exceeds actual pyshical memory of hypervisor.

\textbf{There are two improvements that have been done to overcome this issue:}
\begin{enumerate}
\item Demand paging -> pages are swapped back and forth to alleviate memory requirements of VMs
\item Memory ballooning -> inflated to force the machine to relinquish pages according to it's local memory management policy
\item Exploit redundancy of memory content
\end{enumerate}

\subsubsubsection{Content Based Page Sharing (CPBS)}

One of major feature that has been implemented throughout hypervisors around the world.
Another essential feature of \textbf{CPBS} is \textbf{CPBS} able to perform transparently in the hypervisor layer and doesn't require any modification to guest OS / process.

\textbf{Classified into two categories:}
\begin{enumerate}
\item in-band sharing -> same page detection and merging in the I/O path (mostly disk access path)
\item out-band sharing -> periodically scans memory to identify and merge shareable pages
\end{enumerate}

Out-of-band sharing techniques can potientially take advantage of complete system memory to identify identical memory pages. Out-of-band techniques such as KSM, usually \textbf{oscillate between a sleep period and a scan period}